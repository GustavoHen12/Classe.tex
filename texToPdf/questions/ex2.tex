%HIPOTESE INDUTIVA%
\textbf{Hipótese Indutiva}: Seja $a \in \mathbb{N}$ tal que, para todo $k \in [0..a]$, se $A_{1}, A_{2}, ..., A_{k}$ são conjuntos finitos dois a dois finitos entre si, então
\begin{eqnarray*}
|\bigcup_{i=1}^k A_{i}| = \sum_{i=1}^k |A_{i}|
\end{eqnarray*}
\\
%PASSO INDUÇÃO%
\textbf{Passo Indutivo}:Vamos provar que se $A_{1},A_{2}, ..., A_{a+1}$, são conjuntos finitos distintos entre si, então:
\begin{eqnarray*}
|\bigcup_{i=1}^{a+1} A_{i}| = \sum_{i=1}^{a+1} |A_{i}|
\end{eqnarray*}
Como:
\begin{eqnarray*}
\bigcup_{i=1}^{a+1} A_{i} = (\bigcup_{i=1}^{a} A_{i}) \cup A_{a+1} 
\end{eqnarray*}
Seja $a \geq 2$, pela hipotese intudiva, temos que $|\bigcup_{i=1}^k A_{i}| = \sum_{i=1}^k |A_{i}|$, então:
\begin{eqnarray*}
|\bigcup_{i=1}^{a+1} A_{i}| = |\bigcup_{i=1}^{a} A_{i}| + |A_{a+1}| \\
|\bigcup_{i=1}^{a+1} A_{i}| = |\sum_{i=1}^{a} A_{i}| + |A_{a+1}| \\
|\bigcup_{i=1}^{a+1} A_{i}| = |\sum_{i=1}^{a+1} A_{i}| 
\end{eqnarray*}
\\
%BASE INDUTIVA%
\textbf{Base Indutiva}: